%%%%%%%%%%%%%%%%%%%%%%%%%%%%%%%%%%%%%%%%
% University Assignment Title Page
% LaTeX Template
% Version 1.0 (27/12/12)
%
% This template has been downloaded from:
% http://www.LaTeXTemplates.com
%
% Original author:
% WikiBooks (http://en.wikibooks.org/wiki/LaTeX/Title_Creation)
%
% License:
% CC BY-NC-SA 3.0 (http://creativecommons.org/licenses/by-nc-sa/3.0/)
%
% Instructions for using this template:
% This title page is capable of being compiled as is. This is not useful for
% including it in another document. To do this, you have two options:
%
% 1) Copy/paste everything between \begin{document} and \end{document}
% starting at \begin{titlepage} and paste this into another LaTeX file where you
% want your title page.
% OR
% 2) Remove everything outside the \begin{titlepage} and \end{titlepage} and
% move this file to the same directory as the LaTeX file you wish to add it to.
% Then add \input{./title_page_1.tex} to your LaTeX file where you want your
% title page.
%
%%%%%%%%%%%%%%%%%%%%%%%%%%%%%%%%%%%%%%%%%
%\title{Title page with logo}
%----------------------------------------------------------------------------------------
%	PACKAGES AND OTHER DOCUMENT CONFIGURATIONS
%----------------------------------------------------------------------------------------

\documentclass[11pt]{article}
\usepackage[utf8]{inputenc}
\usepackage{amsmath}
\usepackage{graphicx}
%\usepackage{lmodern}
\usepackage{hyperref}
\usepackage{tabularx}
\usepackage{amsmath}
\usepackage{float}
\usepackage[table]{xcolor}
\usepackage{booktabs}% http://ctan.org/pkg/booktabs

\usepackage{fancyhdr}
\usepackage[hscale=0.75,vscale=0.75]{geometry} % Margin sizes
\usepackage{minted} % for this to work: `sudo apt install python-pygments` and use `-shell-escape` flag with `pdflatex`

\newcommand{\tabitem}{~~\llap{\textbullet}~~}
\newcommand{\HRule}{\rule{\linewidth}{0.5mm}} % Defines a new command for the horizontal lines, change thickness here

\pagestyle{fancy}
\fancyhf{}
\rhead{Mission 2 -- Group G}
\lhead{\textit{LINGI2252}}
\cfoot{\thepage}

\begin{document}
    \begin{titlepage}
        \center % Center everything on the page

        %----------------------------------------------------------------------------------------
        %	HEADING SECTIONS
        %----------------------------------------------------------------------------------------

        \textsc{\LARGE Université Catholique de Louvain }\\[0.8cm] % Name of your university/college
        \includegraphics[scale=0.45]{epl.jpg}
        \\[1.5cm]
        \textsc{\Large LINGI2252}\\[0.5cm] % Major heading such as course name
        \textsc{\large Software Maintenance and Evolution}\\[0.8cm] % Minor heading such as course title

        %----------------------------------------------------------------------------------------
        %	TITLE SECTION
        %----------------------------------------------------------------------------------------

        \HRule \\[0.4cm]
        { \huge \bfseries Mission 2: Improved prototype}\\[0.2cm] % Title of your document
        \HRule \\[1.5cm]

        %----------------------------------------------------------------------------------------
        %	AUTHOR SECTION
        %----------------------------------------------------------------------------------------

		\vfill
        \begin{minipage}{0.4\textwidth}
        \begin{flushleft} \large
        \emph{Authors:}\\
        \textbf{Group G}\\
        \textsc{Gustin}~Simon \\
        1171-14-00\\
        \textsc{Hallet}~Adrien \\
        3276-13-00\\
        \end{flushleft}
        \end{minipage}
        ~
        \begin{minipage}{0.4\textwidth}
        \begin{flushright} \large
        \emph{Professor:} \\
         \textsc{Mens}~Kim \\% Supervisor's Name
         \emph{Assistant:}\\
         \textsc{Duhoux}~Benoît
        \end{flushright}
        \end{minipage}\\[1cm]

        % If you don't want a supervisor, uncomment the two lines below and remove the section above
        %\Large \emph{Author:}\\
        %John~\textsc{Smith}\\[3cm] % Your name

        %----------------------------------------------------------------------------------------
        %	DATE SECTION
        %----------------------------------------------------------------------------------------

        {\large \today}\\[2cm] % Date, change the \today to a set date if you want to be precise

        %----------------------------------------------------------------------------------------
        %	LOGO SECTION
        %----------------------------------------------------------------------------------------

        % Include a department/university logo - this will require the graphicx package

        %----------------------------------------------------------------------------------------

        \vfill % Fill the rest of the page with whitespace
    \end{titlepage}

    \title{AAAAAA}
    \newpage

	\section{Introduction}
		For the course \textit{LINGI2252 -- Software Maintenance and Evolution}, we were asked to improve our first prototype of a house automation system. Specifically, we had to add a parametrization component which would allow us to specify the values and states of the variability points at execution time, a command line interpreter which would allow us to update the state of the house at run time and to use relevant pattern designs to make the code more maintainable.
		
	\section{Parametrization component}
		The parametrization component allows us to easily change the configuration of a house on two different executions without changing the code.
		It does this in a quite straightforward way: when the program executes, it starts by parsing a configuration file. It then creates a house that corresponds to the configuration described in this file.
		
		Notice that we didn't decide to implement the parser as a class of its own but rather to place the parsing logic inside the class \mintinline{java}{House}. As we use an external library to parse the configuration file, we figured out it wouldn't be very interesting to make a parser that would be only a couple of lines long. As can be expected the configuration of the house is then done from within this class itself on basis of the parsed representation of the file. However, if we needed to parse files in other formats, making a class to handle the parsing would be a very good idea.
		
		The configuration is indeed contained within an external file formatted in \textit{JSON}.
		We chose this format for several reasons.
		
		First it is well-known and heavily used as a format for configuration files. This is due to the fact that it is perfectly capable of handling this kind of task. If we had decided to make our own "configuration language", we could have very easily ended up with a syntax that would not fit all possible configurations perfectly, or with a flawed parser that would not work in certain cases.

		Secondly, the fact that it is a well-known format also means that parsers exist in most languages. We didn't have to implement a parser by ourselves or to use one that would be incomplete or buggy.

		Lastly, we are used to this format, which means we can work quickly and efficiently with it. The likelihood of us making an error while using it was thus smaller. This being said, using \textit{YAML} or \textit{TOML} would be perfectly legitimate choices, but we weren't as used to them as we are of \textit{JSON}. Note that using the latter instead of one of those two formats prevents us from having comments in our configuration files, which can be considered a problem.
	
	\section{Command line interpreter}

	\section{Conclusion}

\end{document}
