%%%%%%%%%%%%%%%%%%%%%%%%%%%%%%%%%%%%%%%%
% University Assignment Title Page
% LaTeX Template
% Version 1.0 (27/12/12)
%
% This template has been downloaded from:
% http://www.LaTeXTemplates.com
%
% Original author:
% WikiBooks (http://en.wikibooks.org/wiki/LaTeX/Title_Creation)
%
% License:
% CC BY-NC-SA 3.0 (http://creativecommons.org/licenses/by-nc-sa/3.0/)
%
% Instructions for using this template:
% This title page is capable of being compiled as is. This is not useful for
% including it in another document. To do this, you have two options:
%
% 1) Copy/paste everything between \begin{document} and \end{document}
% starting at \begin{titlepage} and paste this into another LaTeX file where you
% want your title page.
% OR
% 2) Remove everything outside the \begin{titlepage} and \end{titlepage} and
% move this file to the same directory as the LaTeX file you wish to add it to.
% Then add \input{./title_page_1.tex} to your LaTeX file where you want your
% title page.
%
%%%%%%%%%%%%%%%%%%%%%%%%%%%%%%%%%%%%%%%%%
%\title{Title page with logo}
%----------------------------------------------------------------------------------------
%	PACKAGES AND OTHER DOCUMENT CONFIGURATIONS
%----------------------------------------------------------------------------------------

\documentclass[12pt]{article}
\usepackage[utf8]{inputenc}
\usepackage{amsmath}
\usepackage{graphicx}
\usepackage{lmodern}
\usepackage{hyperref}
\usepackage{tabularx}
\usepackage{amsmath}
\usepackage{float}
\usepackage[table]{xcolor}
\usepackage{booktabs}% http://ctan.org/pkg/booktabs

\newcommand{\tabitem}{~~\llap{\textbullet}~~}
\newcommand{\HRule}{\rule{\linewidth}{0.5mm}} % Defines a new command for the horizontal lines, change thickness here

\begin{document}
    \begin{titlepage}
        \center % Center everything on the page

        %----------------------------------------------------------------------------------------
        %	HEADING SECTIONS
        %----------------------------------------------------------------------------------------

        \textsc{\LARGE Université catholique de Louvain }\\[1.5cm] % Name of your university/college
        \includegraphics[scale=0.45]{epl.jpg}
         \\[0.5cm]
        \textsc{\Large LINGI2252}\\[0.5cm] % Major heading such as course name
        \textsc{\large Software Maintenance and Evolution}\\[0.5cm] % Minor heading such as course title

        %----------------------------------------------------------------------------------------
        %	TITLE SECTION
        %----------------------------------------------------------------------------------------

        \HRule \\[0.4cm]
        { \huge \bfseries Home Automation System -- Lab 1}\\[0.4cm] % Title of your document
        \HRule \\[1.5cm]

        %----------------------------------------------------------------------------------------
        %	AUTHOR SECTION
        %----------------------------------------------------------------------------------------

        \begin{minipage}{0.4\textwidth}
        \begin{flushleft} \large
        \emph{Authors:}\\
        \textsc{Gustin} Simon \\
        1171-14-00\\
        \textsc{Hallet} Adrien \\
        3276-13-00\\
        \end{flushleft}
        \end{minipage}
        ~
        \begin{minipage}{0.4\textwidth}
        \begin{flushright} \large
        \emph{Professor:} \\
         \textsc{Mens} Kim \\% Supervisor's Name
         \emph{Assistant:}\\
         \textsc{Duhoux} Benoît
        \end{flushright}
        \end{minipage}\\[1cm]

        % If you don't want a supervisor, uncomment the two lines below and remove the section above
        %\Large \emph{Author:}\\
        %John \textsc{Smith}\\[3cm] % Your name

        %----------------------------------------------------------------------------------------
        %	DATE SECTION
        %----------------------------------------------------------------------------------------

        {\large \today}\\[2cm] % Date, change the \today to a set date if you want to be precise

        %----------------------------------------------------------------------------------------
        %	LOGO SECTION
        %----------------------------------------------------------------------------------------

        % Include a department/university logo - this will require the graphicx package

        %----------------------------------------------------------------------------------------

        \vfill % Fill the rest of the page with whitespace
    \end{titlepage}

    \title{Home Automation System}
    \newpage

    \section{Lexicon}
        \begin{description}
            \item[Domotics] Automation of systems within a house.
            \item[Sensor] Physical device that measures the magnitude of real-life variables.
            \item[Motion sensor] Sensor that detects movements in a given space.
            \item[Proximity sensor] Sensor that detects when a given object is close to it.
            \item[Badge detector] Sensor that detects and checks the rights of a closeby badge.
            \item[Audio sensor] Sensor that measures audio signals.
            \item[Video sensor] Cameras (infrared, normal light, etc).
            \item[Consumption sensor] Sensor that detects and records the consumption of different resources (power, water, etc).
            \item[Temperature sensor] Thermometer.
            \item[Gas sensor] Sensor that detects and measures the quantity of certain substances in the air (smoke, carbon monoxide, etc).
            \item[Clock] Sensor that measures the time flying by (and can be used to set alarms at certain times).
            \item[Heavy appliance] Home systems that use lots of electricity (washing machine, dishwasher, oven, etc).
            \item[IoT] Internet of Things, the use of the internet for appliances like watches, clothes, ...
            \item[Hub] Physical and/or digital system to centralize a group of systems.
        \end{description}

    \section{Features}
        \subsection{Ambiance and well-being}
            \begin{description}
                \item[Temperature management] Turn the heating system on/off.
                \item[Blinds management] Open or close the blinds.
                \item[Lighting management] Turn on/off the lights in different rooms.
                \item[Air quality control] CO detection, humidity sensing, air quality control.
            \end{description}

        \subsection{Safety}
            \begin{description}
                \item[Security control] Detection of potential burglars, alarms, automatic police calls, smart doors (locks and opening).
                \item[Fire handling] Smoke detection, automatic sprinklers, automatic firefighters call.
            \end{description}

        \subsection{Everyday tasks}
            \begin{description}
                \item[Smart kitchen] Manage the cooking appliances, the fridge stocks, ...
                \item[Automatic cleaning] Handling robotics vacuums, lawnmower, swimming pool cleaner.
                \item[Gardening] Humidity captors, automatic sprinklers.
                \item[Heavy appliance] Manage the heavy appliance to minimize electricity consumption and bill (i.e.: night shift).
                \item[Baby and pet care] Tracking position, health and well-being (including food dispenser).
                \item[Sound systems management] Manage the alarms, radios, home-cinema systems.
                \item[Electronics management] Manage computers, televisions, everything that isn't heavy appliance.
                \item[IoT management] Hub to control all the user's IoT equipment.
                \item[Phone management] Redirect calls, send notifications when called, ...
            \end{description}

        \subsection{Consumption control}
            \begin{description}
                \item[Water management] Water consumption notifications, leak detection.
                \item[Power management] Power consumption notifications, real-time management of the appliance currently consuming power.
            \end{description}

\end{document}
