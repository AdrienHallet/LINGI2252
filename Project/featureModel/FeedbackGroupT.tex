%%%%%%%%%%%%%%%%%%%%%%%%%%%%%%%%%%%%%%%%
% University Assignment Title Page
% LaTeX Template
% Version 1.0 (27/12/12)
%
% This template has been downloaded from:
% http://www.LaTeXTemplates.com
%
% Original author:
% WikiBooks (http://en.wikibooks.org/wiki/LaTeX/Title_Creation)
%
% License:
% CC BY-NC-SA 3.0 (http://creativecommons.org/licenses/by-nc-sa/3.0/)
%
% Instructions for using this template:
% This title page is capable of being compiled as is. This is not useful for
% including it in another document. To do this, you have two options:
%
% 1) Copy/paste everything between \begin{document} and \end{document}
% starting at \begin{titlepage} and paste this into another LaTeX file where you
% want your title page.
% OR
% 2) Remove everything outside the \begin{titlepage} and \end{titlepage} and
% move this file to the same directory as the LaTeX file you wish to add it to.
% Then add \input{./title_page_1.tex} to your LaTeX file where you want your
% title page.
%
%%%%%%%%%%%%%%%%%%%%%%%%%%%%%%%%%%%%%%%%%
%\title{Title page with logo}
%----------------------------------------------------------------------------------------
%	PACKAGES AND OTHER DOCUMENT CONFIGURATIONS
%----------------------------------------------------------------------------------------

\documentclass[12pt]{article}
\usepackage[utf8]{inputenc}
\usepackage{amsmath}
\usepackage{graphicx}
\usepackage{lmodern}
\usepackage{hyperref}
\usepackage{tabularx}
\usepackage{amsmath}
\usepackage{float}
\usepackage[table]{xcolor}
\usepackage{booktabs}% http://ctan.org/pkg/booktabs

\newcommand{\tabitem}{~~\llap{\textbullet}~~}
\newcommand{\HRule}{\rule{\linewidth}{0.5mm}} % Defines a new command for the horizontal lines, change thickness here

\begin{document}
    \begin{titlepage}
        \center % Center everything on the page

        %----------------------------------------------------------------------------------------
        %	HEADING SECTIONS
        %----------------------------------------------------------------------------------------

        \textsc{\LARGE Université catholique de Louvain }\\[1.5cm] % Name of your university/college
        \includegraphics[scale=0.45]{epl.jpg}
         \\[0.5cm]
        \textsc{\Large LINGI2252}\\[0.5cm] % Major heading such as course name
        \textsc{\large Software Maintenance and Evolution}\\[0.5cm] % Minor heading such as course title

        %----------------------------------------------------------------------------------------
        %	TITLE SECTION
        %----------------------------------------------------------------------------------------

        \HRule \\[0.4cm]
        { \huge \bfseries Home Automation System -- Lab 1 Review Group T}\\[0.4cm] % Title of your document
        \HRule \\[1.5cm]

        %----------------------------------------------------------------------------------------
        %	AUTHOR SECTION
        %----------------------------------------------------------------------------------------

        \begin{minipage}{0.4\textwidth}
        \begin{flushleft} \large
        \emph{Authors:}\\
        \textsc{Gustin} Simon \\
        1171-14-00\\
        \textsc{Hallet} Adrien \\
        3276-13-00\\
        \end{flushleft}
        \end{minipage}
        ~
        \begin{minipage}{0.4\textwidth}
        \begin{flushright} \large
        \emph{Professor:} \\
         \textsc{Mens} Kim \\% Supervisor's Name
         \emph{Assistant:}\\
         \textsc{Duhoux} Benoît
        \end{flushright}
        \end{minipage}\\[1cm]

        % If you don't want a supervisor, uncomment the two lines below and remove the section above
        %\Large \emph{Author:}\\
        %John \textsc{Smith}\\[3cm] % Your name

        %----------------------------------------------------------------------------------------
        %	DATE SECTION
        %----------------------------------------------------------------------------------------

        {\large \today}\\[2cm] % Date, change the \today to a set date if you want to be precise

        %----------------------------------------------------------------------------------------
        %	LOGO SECTION
        %----------------------------------------------------------------------------------------

        % Include a department/university logo - this will require the graphicx package

        %----------------------------------------------------------------------------------------

        \vfill % Fill the rest of the page with whitespace
    \end{titlepage}

    \title{Feedback Group T's FeatureModel}
    \newpage

    \section{Overall presentation}
        \emph{This is not really pertinent regarding to content, but we still reviewed the visuals of the document.}
        \begin{enumerate}
            \item The lexicon should contain useful definitions. Explaining where the lights are does not fit that requirement.
            \item The lexicon should not explain how an element behaves. You could put it in the scenario or in a legend.
            \item You could rotate (and center) the FeatureModel. We can rotate it manually but I guess not everyone will want/know how to do that.
            \item You forgot the capital letter in \emph{Movementsensor}.

        \end{enumerate}

    \section{Contents}
        \begin{enumerate}
          \item You did not include the required scenario.
          \item The lexicon lacks a proper defintion of the actuator.
          \item If the actuator is something that reacts to ... Then the smoke detector (mandatory) should imply the alarm.
          \item How is the feature "bed" in the bedroom a relevant feature ? Your model/scenario does not explain this.
          \item The \emph{locking mechanism} does not require any lock (although the shutter contol does imply shutters).
          \item The model allows a configuration to select \emph{actuators} by itself but without anything in it. Maybe use an "OR" condition or a mandatory feature inside?
          \item \emph{Camera} is a sensor that is not used anywhere in the model.
          \item The model does not imply that a \emph{ShutterControl} requires a \emph{clock} although you said it in the lexicon. You can use a constraint to state that requirement.
          \item You have \emph{Lights} and \emph{GardenLights}. Do you not consider that \emph{Lights} could include them all?
          \item Your model's biggest subtree is \emph{Home}, but only the \emph{Lights} are in the constraints.
          \item Why are the rooms at the same level as the appliance (lights, shutters, air conditioner) in them? Use subtrees, it makes the model easier to read.
        \end{enumerate}

\end{document}
